\documentclass[norsk]{ifimaster}
\usepackage[utf8]{inputenc}
\usepackage{babel,textcomp,csquotes,ifimasterforside,varioref,graphicx}
\usepackage[T1]{fontenc}
\usepackage[hyphens]{url} % deler lange url-er

% biblatex-pakka med opsjoner
\usepackage[backend=biber,%
%style=numeric,%
backref,%
sortcites,%
defernumbers=true,%
style=authoryear,%
%style=alphabetic%
%sorting=none,% eller nty, nyt o.a
%date=long,% eller short, terse, comp, iso8601
]{biblatex}

% må opplyse om bib-filen
% ved flere bib-filer gjentas kommandoen
\addbibresource{litteraturliste.bib}

% denne kommandoen er for reftex som ikke forstår BibLaTeX
%\bibliography{litteraturliste}%

% noen mulige lokale biblatex tilpasninger
\DefineBibliographyStrings{norsk}{%
   urlseen={Sett: },
   bibliography = {Bibliografi},
   references = {Referanser},
   editor = {redaktør},
   translator={oversetter},
   %page={side},
   %pages={sidene},
   and={og},
}

\DeclareFieldFormat{url}{\url{#1}} % fjerner hardkodet "URL: " foran url

\DeclareUrlCommand\url{\def\UrlLeft{\newline}\def\UrlRight{\newline}%
\urlstyle{sf}} % setter inn passende linjeskift

% biblatex anbefaler at hyperref blir lastet inn etter biblatex
\usepackage{hyperref}

\title{Design mønstre:}
\subtitle{ Klassisk mot prototypisk programdesign av en nettassistent  }
\subsubtitle{---om ``design patterns'' og implementasjon i javascript}
% Sett inn ditt eget navn her:
\author{Johannes Akse}

\begin{document}
\maketitle{}

\tableofcontents
\nocite{*}
\section{Introduksjon}
\subsection{Bakgrunn}
Dette essayet skal handle om ``design patterns'' og hvordan disse kan implementeres i javascript. 
\paragraf



\section{Avgrensning og problemstilling}
\section{Disposisjon}





\section{Metode (deltakerene)}



\section{Diskusjon (Konflikten tilspisses)}



\section{Avsluting}

 





\section{User interface management systems}

A user interface management system (UIMS) is a software 
component that is separate from the application program 
that performs the underlaying task **Olsen, 1992**.

\newpage
% overskriften på referanselista (dokumentklassen article)
\renewcommand{\refname}{Litteraturliste}

% redefiner \bibname ved bruk av dokumentklassen book

% Litteraturlista inn i innholdsfortegnelsen
\addcontentsline{toc}{section}{\refname}

\printbibheading
\printbibliography[type=book, title={Bøker}]
\printbibliography[type=article, title={Artikler}]
\printbibliography[type=manual, title={Manualer}]
\printbibliography[nottype=book, nottype=article,%
nottype=manual, title={Øvrige dokumenter}]

\end{document}
